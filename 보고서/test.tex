\documentclass{article}
\usepackage{kotex}
\usepackage{array}
\usepackage{geometry}

% 페이지 설정
\geometry{a4paper, margin=1in}

\begin{document}

\title{리눅스 주통기 취약점 점검 결과 보고서}
\author{작성자: 오나희}
\date{} % 날짜를 빈 상태로 설정

\maketitle

\section{파일 및 디렉토리 관리 점검 결과}

\begin{table}[h!]
\centering
\begin{tabular}{|>{\raggedright\arraybackslash}p{3cm}|>{\raggedright\arraybackslash}p{5cm}|>{\raggedright\arraybackslash}p{5cm}|>{\raggedright\arraybackslash}p{3cm}|}
\hline
\textbf{이름(중요도)} & \textbf{점검내용} & \textbf{점검목적} & \textbf{점검결과} \\
\hline
U-05 (상) & root 홈, 패스 디렉터리 권한 및 패스 설정 & 불법적으로 생성한 디렉터리 및 명령어를 우선으로 실행되지 않도록 설정하기 위해 & . 포함하고 있지 않음 \\
\hline
 &  &  &  \\
\hline
 &  &  &  \\
\hline
\end{tabular}
\caption{점검 결과 요약}
\end{table}

\end{document}
